\documentclass{article}

\usepackage[margin=1.1in]{geometry}
\usepackage{graphicx}
\usepackage{indentfirst}

\begin{document}

\title{Participatory Scientific Peer Review}
\author{Lucas Burns}

\maketitle

This is an outline for a peer review model, shown in Figure \ref{fig:peer}, in which scientists participate directly with one another openly in the technical review and dissemination of research pre-publication. For comparison, the traditional peer review process is shown in Figure \ref{fig:traditional}.

Scientists experience high pressure to publish their work. It is effectively the only way they have to demonstrate the value of their research when seeking an academic position or funding. This feeds a culture of ``publish or perish'' in academia. 

Journals are high stakes. A scientist's work is either accepted or rejected on the basis of a small number of peer reviews, the content of which is undisclosed and unavailable to public evaluation. The non-disclosure of peer reviews omits any evidence for or against the value of a scientist's research.

Fortunately, the service that journals provide scientists, namely validation and dissemination of their work, is largely a service that scientists are capable of providing to each other directly.

The weight carried by an article's peer reviews is what is ultimately used by a journal to validate or invalidate a scientist's work, and scientists already do this work for journals free of charge! 

Furthermore, if scientists simply commit to openly publish their reviews, then these reviews, and perhaps derived metrics, can be cited directly in efforts to demonstrate the value of one's own research when seeking a job or funding.

% Journal editors ask scientists to not only play the role of a technical evaluator, but also to make publication recommendations. As a result, a small number of reviewers have disproportionate influence over the publication decisions made by editors, which results in conflicts of interest and even fraud.

% Figure \ref{fig:peer} shows a flow diagram of the proposed process, in which peer review and publishing have been decoupled from on another.

\begin{figure}[!h]
  \centering
  \includegraphics[width=1\textwidth]{peer}
  \caption{Flow diagram of the proposed peer review and publishing model. Under this model, peer reviews are decoupled from the publication process and published openly. Advantages to this model include (1) the elimination of gatekeeping in the traditional publishing process, (2) published and citable technical peer reviews that are open to public evaluation, (3) peer to peer screening and dissemination of content, (4) the use of reviews as an indicator of the possible value of research, and (5) compatibility with multiple publishing models, including independent, overlay, and traditional publishing.\label{fig:peer}}
\end{figure}

\begin{figure}[!h]
  \centering
  \includegraphics[width=0.65\textwidth]{traditional}
  \caption{Flow diagram of the traditional peer review and publishing process.\label{fig:traditional}}
\end{figure}

% In the interest of minimizing conflicts of interest with the mission of science, it is important that an implementation of this model be open and shared infrastructure, not a service that relies on trust of a party that benefits financially from servicing.

% I don't want to build a github, I want to build an smtp.

% Use of verifiable data structures to decentralize data storage. Have infrastructure for private binary feeds and archives, but not for publicly appendable data structures.

\end{document}
